% Options for packages loaded elsewhere
\PassOptionsToPackage{unicode}{hyperref}
\PassOptionsToPackage{hyphens}{url}
%
\documentclass[
]{article}
\usepackage{lmodern}
\usepackage{amssymb,amsmath}
\usepackage{ifxetex,ifluatex}
\ifnum 0\ifxetex 1\fi\ifluatex 1\fi=0 % if pdftex
  \usepackage[T1]{fontenc}
  \usepackage[utf8]{inputenc}
  \usepackage{textcomp} % provide euro and other symbols
\else % if luatex or xetex
  \usepackage{unicode-math}
  \defaultfontfeatures{Scale=MatchLowercase}
  \defaultfontfeatures[\rmfamily]{Ligatures=TeX,Scale=1}
\fi
% Use upquote if available, for straight quotes in verbatim environments
\IfFileExists{upquote.sty}{\usepackage{upquote}}{}
\IfFileExists{microtype.sty}{% use microtype if available
  \usepackage[]{microtype}
  \UseMicrotypeSet[protrusion]{basicmath} % disable protrusion for tt fonts
}{}
\makeatletter
\@ifundefined{KOMAClassName}{% if non-KOMA class
  \IfFileExists{parskip.sty}{%
    \usepackage{parskip}
  }{% else
    \setlength{\parindent}{0pt}
    \setlength{\parskip}{6pt plus 2pt minus 1pt}}
}{% if KOMA class
  \KOMAoptions{parskip=half}}
\makeatother
\usepackage{xcolor}
\IfFileExists{xurl.sty}{\usepackage{xurl}}{} % add URL line breaks if available
\IfFileExists{bookmark.sty}{\usepackage{bookmark}}{\usepackage{hyperref}}
\hypersetup{
  hidelinks,
  pdfcreator={LaTeX via pandoc}}
\urlstyle{same} % disable monospaced font for URLs
\usepackage{color}
\usepackage{fancyvrb}
\newcommand{\VerbBar}{|}
\newcommand{\VERB}{\Verb[commandchars=\\\{\}]}
\DefineVerbatimEnvironment{Highlighting}{Verbatim}{commandchars=\\\{\}}
% Add ',fontsize=\small' for more characters per line
\newenvironment{Shaded}{}{}
\newcommand{\AlertTok}[1]{\textcolor[rgb]{1.00,0.00,0.00}{\textbf{#1}}}
\newcommand{\AnnotationTok}[1]{\textcolor[rgb]{0.38,0.63,0.69}{\textbf{\textit{#1}}}}
\newcommand{\AttributeTok}[1]{\textcolor[rgb]{0.49,0.56,0.16}{#1}}
\newcommand{\BaseNTok}[1]{\textcolor[rgb]{0.25,0.63,0.44}{#1}}
\newcommand{\BuiltInTok}[1]{#1}
\newcommand{\CharTok}[1]{\textcolor[rgb]{0.25,0.44,0.63}{#1}}
\newcommand{\CommentTok}[1]{\textcolor[rgb]{0.38,0.63,0.69}{\textit{#1}}}
\newcommand{\CommentVarTok}[1]{\textcolor[rgb]{0.38,0.63,0.69}{\textbf{\textit{#1}}}}
\newcommand{\ConstantTok}[1]{\textcolor[rgb]{0.53,0.00,0.00}{#1}}
\newcommand{\ControlFlowTok}[1]{\textcolor[rgb]{0.00,0.44,0.13}{\textbf{#1}}}
\newcommand{\DataTypeTok}[1]{\textcolor[rgb]{0.56,0.13,0.00}{#1}}
\newcommand{\DecValTok}[1]{\textcolor[rgb]{0.25,0.63,0.44}{#1}}
\newcommand{\DocumentationTok}[1]{\textcolor[rgb]{0.73,0.13,0.13}{\textit{#1}}}
\newcommand{\ErrorTok}[1]{\textcolor[rgb]{1.00,0.00,0.00}{\textbf{#1}}}
\newcommand{\ExtensionTok}[1]{#1}
\newcommand{\FloatTok}[1]{\textcolor[rgb]{0.25,0.63,0.44}{#1}}
\newcommand{\FunctionTok}[1]{\textcolor[rgb]{0.02,0.16,0.49}{#1}}
\newcommand{\ImportTok}[1]{#1}
\newcommand{\InformationTok}[1]{\textcolor[rgb]{0.38,0.63,0.69}{\textbf{\textit{#1}}}}
\newcommand{\KeywordTok}[1]{\textcolor[rgb]{0.00,0.44,0.13}{\textbf{#1}}}
\newcommand{\NormalTok}[1]{#1}
\newcommand{\OperatorTok}[1]{\textcolor[rgb]{0.40,0.40,0.40}{#1}}
\newcommand{\OtherTok}[1]{\textcolor[rgb]{0.00,0.44,0.13}{#1}}
\newcommand{\PreprocessorTok}[1]{\textcolor[rgb]{0.74,0.48,0.00}{#1}}
\newcommand{\RegionMarkerTok}[1]{#1}
\newcommand{\SpecialCharTok}[1]{\textcolor[rgb]{0.25,0.44,0.63}{#1}}
\newcommand{\SpecialStringTok}[1]{\textcolor[rgb]{0.73,0.40,0.53}{#1}}
\newcommand{\StringTok}[1]{\textcolor[rgb]{0.25,0.44,0.63}{#1}}
\newcommand{\VariableTok}[1]{\textcolor[rgb]{0.10,0.09,0.49}{#1}}
\newcommand{\VerbatimStringTok}[1]{\textcolor[rgb]{0.25,0.44,0.63}{#1}}
\newcommand{\WarningTok}[1]{\textcolor[rgb]{0.38,0.63,0.69}{\textbf{\textit{#1}}}}
\setlength{\emergencystretch}{3em} % prevent overfull lines
\providecommand{\tightlist}{%
  \setlength{\itemsep}{0pt}\setlength{\parskip}{0pt}}
\setcounter{secnumdepth}{-\maxdimen} % remove section numbering

\author{}
\date{}

\begin{document}

\hypertarget{lightning-mcqueen}{%
\section{Lightning McQueen}\label{lightning-mcqueen}}

Lighting McQueen is our phases of models created for arm gesture
recognition. Bellow you will find the setup process for beeing able to
run the code used for machine learning part of this project.

\hypertarget{requirements}{%
\subsection{Requirements}\label{requirements}}

\begin{itemize}
\item
  Python 3.6 or 3.7
\item
  We recommend using PyCharm IDE (https://www.jetbrains.com/pycharm/)
\end{itemize}

\hypertarget{configure-and-activate-virtual-environment}{%
\subsection{Configure and activate virtual
environment}\label{configure-and-activate-virtual-environment}}

\hypertarget{pycharm}{%
\subsubsection{PyCharm}\label{pycharm}}

If a virtual environment was not created when setting up a new project
in PyCharm follow JetBrains guide on setting up a virtual environment
(https://www.jetbrains.com/help/pycharm/creating-virtual-environment.html)

\hypertarget{terminal}{%
\subsubsection{Terminal}\label{terminal}}

\hypertarget{check-to-see-if-your-install-of-python-has-pip}{%
\paragraph{Check to see if your install of Python has
pip}\label{check-to-see-if-your-install-of-python-has-pip}}

\begin{Shaded}
\begin{Highlighting}[]
\ExtensionTok{pip}\NormalTok{ {-}h}
\end{Highlighting}
\end{Shaded}

\hypertarget{install-virtualenv-package}{%
\paragraph{Install virtualenv
package}\label{install-virtualenv-package}}

\begin{Shaded}
\begin{Highlighting}[]
\ExtensionTok{pip}\NormalTok{ install virtualenv}
\end{Highlighting}
\end{Shaded}

\hypertarget{create-virtual-environment}{%
\paragraph{Create virtual
environment}\label{create-virtual-environment}}

Make sure to be in the project directory.

\begin{Shaded}
\begin{Highlighting}[]
\ExtensionTok{virtualenv}\NormalTok{ venv}
\end{Highlighting}
\end{Shaded}

\hypertarget{activate-virtual-environment}{%
\paragraph{Activate virtual
environment}\label{activate-virtual-environment}}

Mac OS and Linux

\begin{Shaded}
\begin{Highlighting}[]
\BuiltInTok{source}\NormalTok{ venv/bin/activate}
\end{Highlighting}
\end{Shaded}

Windows

\begin{Shaded}
\begin{Highlighting}[]
\ExtensionTok{venv}\NormalTok{\textbackslash{}Scripts\textbackslash{}activate}
\end{Highlighting}
\end{Shaded}

If you want to deactivate the virtual environment simply run
\texttt{deactivate} in the terminal window.

\hypertarget{installation-of-needed-packages}{%
\subsection{Installation of needed
packages}\label{installation-of-needed-packages}}

Use the package manager \href{https://pip.pypa.io/en/stable/}{pip} for
installing needed packages.

\begin{Shaded}
\begin{Highlighting}[]
\ExtensionTok{pip}\NormalTok{ install plaidml{-}keras plaidbench}
\ExtensionTok{pip}\NormalTok{ install {-}U matplotlib}
\ExtensionTok{pip}\NormalTok{ install opencv{-}python}
\ExtensionTok{pip}\NormalTok{ install tqdm}
\ExtensionTok{pip}\NormalTok{ install termcolor}
\ExtensionTok{pip}\NormalTok{ install sty}
\ExtensionTok{pip}\NormalTok{ install pickle}
\end{Highlighting}
\end{Shaded}

\hypertarget{setup-plaidml}{%
\subsubsection{Setup PlaidML}\label{setup-plaidml}}

\begin{Shaded}
\begin{Highlighting}[]
\VariableTok{+ If you have a dedicated GPU its recomended to select it as your accelerator.}
\end{Highlighting}
\end{Shaded}

Choose which accelerator you'd like to use (many computers, especially
laptops, have multiple) In the terminal of your python project (venv)
write:

\begin{Shaded}
\begin{Highlighting}[]
\ExtensionTok{plaidml{-}setup}
\end{Highlighting}
\end{Shaded}

\begin{itemize}
\tightlist
\item
  Enable experimental mode
\item
  Select your accelerator
\end{itemize}

Now try benchmarking MobileNet:

\begin{Shaded}
\begin{Highlighting}[]
\ExtensionTok{plaidbench}\NormalTok{ keras mobilenet}
\end{Highlighting}
\end{Shaded}

\begin{itemize}
\tightlist
\item
  You are now good to go!
\end{itemize}

\end{document}
